\documentclass[german,12pt]{article}


\author{Michael Klopsch}

\usepackage{babel}
\usepackage{xcolor}
\usepackage[hidelinks]{hyperref}
\hypersetup{linktoc=all}
\usepackage{graphicx}
\usepackage{enumitem}
\usepackage[titles]{tocloft}

\addtolength{\cftsubsecnumwidth}{10pt}  % with tocloft, https://tex.stackexchange.com/questions/296545

\widowpenalties 1 10000  % https://tex.stackexchange.com/questions/21983

\makeatletter % https://tex.stackexchange.com/questions/8351/what-do-makeatletter-and-makeatother-do
\renewcommand*{\fps@figure}{hbtp} % https://tex.stackexchange.com/questions/364817/how-to-change-default-float-options-for-figures
\g@addto@macro\@floatboxreset\centering % https://tex.stackexchange.com/questions/2651/should-i-use-center-or-centering-for-figures-and-tables
\makeatother

\addtolength{\topmargin}{-0.8in}
\addtolength{\textheight}{1in}

\renewcommand{\thesection}{\Roman{section}}
\renewcommand{\thesubsubsection}{\thesubsection.\Alph{subsubsection}}



\newcommand{\spectitle}[1]{\title{Nutzterdokumentation -- Buchschloss -- #1}}
\newcommand{\linkandref}[2]{\hyperref[#1]{#2}\footnote{Siehe Abs. \ref{#1} auf S. \pageref{#1}, }}
\newcommand{\hinweis}[1]{\begin{quote}\textbf{Hinweis}: #1\end{quote}}
\newcommand{\figref}[1]{Abb. \ref{#1}}


\spectitle{Textoberfläche (cli)}

\newcommand{\command}[1]{\mbox{\texttt{#1}}}

\begin{document}

\maketitle
\newpage

\tableofcontents
\newpage

\section{Starten der Textoberfläche und erste Schritte}
\label{sec:launch_and_start}

\subsection{Starten der Textoberfläche}
\label{subsec:launch_and_start:launch}

Zum Starten der Textoberfläche wird der Terminal geöffnet. Man sollte sich in der sog. ``home-directory'' befinden. Ist das nicht so, kann man mit \mbox{\command{\$ cd /home/sgb}\footnote{\texttt{\$} repräsentiert die Eingabeaufforderung und sollte nicht getippt werden}} dorthin gelangen. Dann kann man mittels \command{\$ ./launcher.py cli} die Oberfläche starten. Für weitere Optionen kann \command{\$ ./launcher.py --help} verwendet werden.

Bei Start der Textoberfläche erscheint ein kurzer Begrü\ss ungstext sowie die Eingabeaufforderung \command{----- > } (im Folgenden verkürzt durch \texttt{>} dargestellt). Jetzt können Befehle eingetippt werden.

\subsection{Befehlsaufbau}
\label{subsec:launch_and_start:command_syntax_intro}

Um etwas zu tun, werden Befehle eingegeben, wie z.B.~``Erstelle ein neues Buch''. Jeder dieser Befehle hat einen speziellen Namen (in Englisch, z.B. \verb|new\_book| für ``Erstelle ein neues Buch'') und einige nehmen sog.~Parameter, also zusätzliche Informationen (z.B. ``Autor'', wenn man ein neues Buch erstellt).

Der Befehl und einzelne Parameter werden durch Leerzeichen getrennt. Befindet sich innerhalb eines Parameters ein Leerzeichen, muss der gesamte Parameter in Anführungszeichen umschlossen werden.

Jeder Parameter hat einen Namen und wird nach dem Befehl oder ggf. anderen Parametern angegeben. Dabei wird erst der Name, dann ein Gleichheitszeichen und zuletzt der Parameterwert geschrieben

%\hinweis{``Befehl'' kann sich je nach Kontext auf einen Befehl mit oder ohne Parameter beziehen}  % ist das noch aktuell?

\subsection{Der \texttt{help}-Befehl}
\label{subsec:launch_and_start:help}

Mittels des \texttt{help}-Befehls kann man sich Hilfe anzeigen lassen. Wenn man nur \command{> help} ohne Parameter eingibt, wird allgemeine Hilfe angezeigt. Gibt man \mbox{\texttt{> help commands}} an, werden grundlegene Informationen zu allen Befehlen angezeigt. Mit \command{> help (Befehl)} (``(Befehl)'' durch einen tatsächlichen Befehl ersetzt) werden erweiterte Informationen zum Befehl angezeigt.

\section{Genaue Regeln für Befehle}
\label{sec:command_syntax_complete}

In Abschnitt \ref{sec:launch_and_start} ist bereits kurz erläutert worden, wie Befehle aufgebaut werden. Hier wird nun detailliert erklärt, was genau dabei möglich ist.

\subsection{Verben und parameter}
\label{subsec:command_syntax_complete:verbs_and_params}

Um einen Befehl einzugeben, braucht man immer das sog. ``Befehlsverb'', welches den Befehl eindeutig identifiziert, wie z.B. \verb+new\_book+, um ein neues Buch zu erstellen.
Je nach Befehl müssen und/oder können dann Parameter angegeben werden. Jeder Parameter hat einen Namen und eine Position. In einigen Fällen kann es auch Parameter geben, die nur mittels Namen oder mittels Position angegeben werden können; dies wird dann gesondert angegeben.

Generell kann ein Parameter entweder durch Position oder durch Namen angegeben werden. Die Angaben verschiedenen Parameter können auch gemischt erfolgen. Um einen Parameter per Position anzugeben, wird der Wert direkt hinter das Befehlsverb bzw. den vorangehenden Parameter gesetzt. So können beliebig viele Parameter per Position angegeben werden. Um einen Parameter per Namen anzugeben, wird zunächst der Name des Parameters geschrieben, dann, ohne Leerzeichen, ein Gleichheitszeichen und am Ende der Wert des Parameters.

Das Befehlsverb sowie einzelnen Parameter untereinander werden mit Leerzeichen getrennt. Ist in einem Parameter ein Leerzeichen vorhanden, muss der gesamte Parameter in Anführungszeichen umschlossen werden.

\hinweis{Aus technischen Gründen ist es derzeit nicht möglich, einen Parameter, der ein Gleichheitszeichen im Wert hat, per Namen anzugeben.}

\subsection{Parameterwerte}
\label{subsec:command_syntax_complete:param_values}

Als Wert für einen Parameter kann entweder eine Variable (s. \ref{subsec:command_syntax_complete:variables} oder einer der hier aufgeführten Werte verwendet werden.

Es gibt verschiedene Arten von Parameterwerten:

\begin{itemize}
\item[Text]  Text wird von Apostrophen umschlossen. Wenn der Text nich mit einem anderen Wert verwechselt werden kann, dürfen diese auch weggelassen werden.
\item[Zahl] angegebene Zahlen können entweder ganze Zahlen oder Kommazahlen, wobei statt eines Kommas ein Punkt verwendet wird, sein. Zahlen dürfen keine Leerzeichen\footnote{oder andere Nicht-Ziffern} enthalten.
\item[Wahrheitswert] entweder ``True'' oder ``False''
\item[Datum] ein Datum kann im \texttt{YYYY-MM-TT}-Format angegeben werden.
\item[Liste] Listen enthalten mehrere Elemente, die allesamt gültige Paramterwerte\footnote{keine Variablen} sein müssen. Eine Liste beginnt mit einer öffnenden Klammer und endet mit einer schlie\ss enden Klammer. Die Elemente in Listen dürfen keine Variablen sein und Text in Listen muss immer in Apostrophen umschlossen sein.
\item[Zuordnung] Eine Zuordnung ordnet einem Wert einen anderen zu. Zuordnungen beginnen mit einer geöffneten geschweiften Klammer, enthalten beliebig viele mit Kommata getrennte Einträge und enden mit einer Schlie\ss enden geschweiften Klammer. Ein Eintrag besteht aus dem Schlüssel, gefolgt von einem Doppelpunkt und dem Wert. Schlüssel und Werte können beliebig sein, jedoch müssen Texte mit Apostrophen umschlossen sein und Variablen dürfen nicht verwendet werden. Zusätzlich darf eine Zuordnung nicht als Schlüssel verwendet werden.
\end{itemize}

\hinweis{Die Angeführten sollten zwar alle Fälle abdecken, sind aber nicht alle der unterstützten Werte. Derzeit werden alle Python-Literalwerte akzeptiert.}

\subsection{Meta-Optionen}
\label{subsec:command_syntax_complete:meta_options}

Meta-Optionen werden nach den Parametern angegeben. Sie bestehen aus dem Namen und ggf. einem nach einem Leerzeichen folgenden Wert. Derzeit gibt es zwei Optionen:

\begin{enumerate}
\item \texttt{--store} speichert den vom Befehl gegebenen Wert in der Variablen (s. \ref{subsec:command_syntax_complete:variables}), deren Name nach \texttt{--store} angegeben wird
\item \texttt{--cmd} oder \texttt{-c} führt den Befehl mit dem Namen, der nach \texttt{--cmd} bzw. \texttt{-c} angegeben wird aus, wobei der vom Hauptbefehl zurückgegebene Wert als Parameter mit Position 1 verwendet wird. Diese Option eignet sich besonders, um sich Suchergebnisse anzeigen zu lassen.
\end{enumerate}

\subsection{Variablen}
\label{subsec:command_syntax_complete:variables}

Variablen werden mit der \texttt{--store} Meta-Option erstellt. Sie können als Parameterwert verwendet werden. Dabei wird der Name in spitzen Klammern (\texttt{<} und \texttt{>}) eingeschlossen. Die spezielle Variable \texttt{<>} (also ohne Namen) repräsentiert immer das letzte Ergebnis\footnote{Ausnahme: Beim \texttt{foreach}-Befehl, siehe dazu \ref{subsec:command_list:foreach}}.


\section{Befehlsliste}
\label{sec:command_list}

Hier werden alle Befehle mit ihren Parametern aufgeführt.

Die Meisten Beschreibungen beginnt mit dem Text, der von \command{> help} angezeigt wird und listet dann alle Parameter samt möglichen Werten auf.

%\subsection{\texttt{command}}
%\label{subsec:command_list:command}
%
%\begin{verbatim}
%descr
%\end{verbatim}
%
%Parameter:
%\begin{tabular}{|p{0.2\textwidth}|p{0.2\textwidth}|p{0.6\textwidth}|}
%
%\end{tabular}

\subsection{\texttt{help} -- Hilfe}
\label{subsec:command_list:help}

\begin{verbatim}
help(name=None)
    Display help for an action.
    If "commands" is passed, list possible commands.
    If no action is given, display general help.
\end{verbatim}

Als Parameter wird entweder das Befehlsverb, zu dem Hilfe angezeigt werden soll, ``commands'', um alle verfügbaren Befehle mit kurzer Erläuterung anzuzeigen oder nichts, um generelle Hilfe anzuzeigen genommen.


\subsection{\texttt{login} -- Einloggen}
\label{subsec:command_list:login}

\begin{verbatim}
login(name: str, password: str)
    attempt to login the Member with the given name and password.

    Try all iterations specified in config.HASH_ITERATIONS and update to newest (first) one where applicable
    raise a BuchSchlossError on failure
\end{verbatim}

Parameter:

\begin{tabular}{|p{0.2\textwidth}|p{0.2\textwidth}|p{0.6\textwidth}|}
Name & Wertart & Bemerkung \\
\hline
name & Text & Der Name des Mitglieds, das eingeloggt werden soll.
\end{tabular}

Das Passwort wird \textbf{danach auf Aufforderung} eingegeben; die eingegebenen Zeichen werden nicht auf dem Bildschirm gezeigt.

\subsection{\texttt{logout} -- Ausloggen}
\label{subsec:command_list:logout}

\begin{verbatim}
logout()
    log the currently logged in Member out
\end{verbatim}

keine Parameter

Loggt das derzeit eingeloggte Mitglied aus.

\subsection{\texttt{new\_person} -- Neue Schüler erstellen}
\label{subsec:command_list:new_person}

\begin{verbatim}
new_person(id: int, first_name: str, last_name: str, class_: str, 
max_borrow: int = 3, libraries: Iterable[str] = (), pay: bool = None, 
pay_date: datetime.date = None)
    Attempt to create a new Person with the given kwargs.

    raise BuchSchlossError on failure.

    See Person for details on arguments
    If ``pay`` is True and ``pay_date`` is None,
        set the pay_date to ``datetime.date.today()``
\end{verbatim}

Parameter:

\begin{tabular}{|p{0.2\textwidth}|p{0.2\textwidth}|p{0.6\textwidth}|}
Name & Wertart & Bemerkung\\
\hline
id & ganze Zahl & Die Schülerausweisnummer des neuen Schülers \\
first\_name & Text & Der Vorname des neuen Schülers  \\
last\_name & Text & Der Nachname des neuen Schülers \\
class\_ & Text & Die Klasse des neuen Schülers \\
max\_borrow & ganze Zahl &  wie viele Bücher der neue Schüler maximal ausleihen darf (3, falls nicht angegeben) \\
libraries & Liste aus Texten & aus welchen Bibliotheken der Schüler ausleihen darf (``main'', falls nicht angegeben) \\
pay & Wahrheitswert & wenn dies wahr ist und ``pay\_date'' nicht angegeben wird, wird das Bezahldatum auf den aktuellen Tag gesetzt \\
pay\_date & Datum & das Datum, an dem der neue Schüler bezahlt hat (weglassen, falls nicht bezahlt wurde) 
\end{tabular}

\subsection{\texttt{new\_book} -- Neue Bücher erstellen}
\label{subsec:command_list:new_book}

\begin{verbatim}
new_book(isbn: int = None, year: int = None, groups: Iterable[str] = (),
library: str = 'main', **kwargs: str) -> int
    Attempt to create a new Book with the given kwargs and return the ID

    automatically create groups as needed
    raise BuchSchlossError on failure

    See Book.__doc__ for details on arguments
\end{verbatim}

Parameter:

\begin{tabular}{|p{0.2\textwidth}|p{0.2\textwidth}|p{0.5\textwidth}|p{0.1\textwidth}}
Name & Wertart & Bemerkung & benötigt\\
\hline
isbn & ganze Zahl & Die ISBN des Buches, bitte im ISBN-13-Format & Ja\\
year & ganze Zahl & Das Erscheinungsjahr des Buches & Nein \\
groups & Liste aus Texten & die Gruppen, in die das Buch kommen soll & Nein\\
library & Text & Name der Bibliothek, in die das Buch kommen soll & Nein \\
\hline
\end{tabular}

\begin{tabular}{|c|}folgende Parameter können nur mit Namen angegeben werden: \end{tabular}

\begin{tabular}{|p{0.2\textwidth}|p{0.2\textwidth}|p{0.5\textwidth}|p{0.1\textwidth}}
\hline
author & Text & Der Autor des Buchs & Ja \\
title & Text & Der Titel des Buchs & Ja \\
series & Text & Die Serie des Buchs & Nein \\
language & Text & Die Sprache des Buchs & Ja \\
concerned\_\-people & Text & weitere mit dem Buch zusammenhängende Menschen & Nein \\
medium & Text & Das Medium des Buchs & Ja \\
genres & Text & Die Genres des Buchs & Nein \\
shelf & Text & Das Regal des Buchs & Ja \\
\end{tabular}

\subsection{\texttt{new\_library} -- Neue Bibliotheken erstelen}
\label{subsec:command_list:new_library}

\begin{verbatim}
new_library(name: str, books: Iterable[int], people: Iterable[int], 
pay_required: bool = True)
    Create a new Library with the specified name and add it to the specified
        people and books.

    ``people`` and ``books`` are iterables of the IDs of the poeple and books
        to gain access / be transferred to the new library
    if a Library with the given name already exists, add it to the given
        people and books
    return a set of strings describing any encountered errors
\end{verbatim}

Parameter:

\begin{tabular}{|p{0.2\textwidth}|p{0.2\textwidth}|p{0.6\textwidth}|}
Name & Wertart & Bemerkung\\
\hline
name & Text & der Name der neuen Bibliothek \\
books & Liste aus ganzen Zahlen & die IDs der Bücher, die in die neue Bibliothek wechseln sollen \\
people & List aus ganzen Zahlen & die Schülerausweisnummern der Schüler, die Zugang zur neuen Bibliothek erhalten sollen \\
pay\_required & Wahrheitswert & ob Schüler bezahlt haben müssen, um Bücher aus der Bibliothek auszuleihen
\end{tabular}

\subsection{\texttt{new\_group} -- Neue Gruppen erstellen}
\label{subsec:command_list:new_group}

\begin{verbatim}
new_group(name: str, books: Iterable[int])
    Create a new Group with the specified name and add the given books to it

    ``books`` is an iterable of the IDs of the books to be added
    return a set of strings describing encountered errors
\end{verbatim}

Parameter:

\begin{tabular}{|p{0.2\textwidth}|p{0.2\textwidth}|p{0.6\textwidth}|}
Name & Wertart & Bemerkung\\
\hline
name & Text & der Name der neuen Gruppe\\
books & Liste aus ganzen Zahlen & die IDs der Bücher, die der neuen Gruppe hinzugefügt werden sollen  \\
\end{tabular}

\subsection{\texttt{new\_member} -- Neue Mitglieder erstellen}
\label{subsec:command_list:new_member}

\begin{verbatim}
new_member(name: str, password: str, level: int)
    Create a new Member with the specified properties.

        A salt of the length specified in config.HASH_SALT_LENGTH is randomly generated and used to hash the password.
        See the wrapping pbkdf function defined in this module for hashing details

    This function requires authentication in form of
    a `current_password` argument containing the currently
    logged in member's password
\end{verbatim}

Parameter:

\begin{tabular}{|p{0.2\textwidth}|p{0.2\textwidth}|p{0.6\textwidth}|}
Name & Wertart & Bemerkung\\
\hline
name & Text & der Name des neuen Mitglieds\\
level & ganze Zahl & Das Level, welches das neue Mitglied haben soll (0=inaktiv, 1=Mitarbeiter, 2=Supermitarbeiter, 3=Administrator, 4=Superadministrator)  \\
\end{tabular}

Das Passwort des derzeit eingeloggten Mitglieds sowie das Passwort, welches das neue Mitglied haben soll, werden \textbf{danach auf Aufforderung} eingegeben.

\subsection{\texttt{edit\_person} -- Schüler verwalten}
\label{subsec:command_list:edit_person}

\begin{verbatim}
edit_person(person: Union[int, v2.models.Person], **kwargs)
    Edit a Person based on the arguments given.

    See Person.__doc__ for more information on the arguments
    `pay` may be passed as argument with a truthy value to set
        `pay_date` to `datetime.date.today()`

    raise a BuchSchlossError if the Person isn't found.
    Return a set of errors found during updating the person's libraries
\end{verbatim}

Als Parameter werden die Schülerausweisnummer des zu verwaltenden Schülers (erste Stelle oder ``person'') und, nur mit Namen, die Parameter, die auch für \command{new\_person} (\linkandref{subsec:command\_list:new\_person}) genommen. Es müssen nur die Daten angegeben werden, die verändert werden sollen.

\subsection{\texttt{edit\_book} -- Bücher verwalten}
\label{subsec:command_list:edit_book}

\begin{verbatim}
edit_book(book: Union[int, v2.models.Book], **kwargs)
    Edit a Book based on the arguments given. 

    See Book.__doc__ for more information on the arguments
    raise a BuchSchlossError if the Book isn't found.
\end{verbatim}

Als Parameter werden die Schülerausweisnummer des zu verwaltenden Buchs (erste Stelle oder ``book'') und, nur mit Namen, die Parameter, die auch für \command{new\_book} (s. \ref{subsec:command_list:new_book}) genommen. Es müssen nur die Daten angegeben werden, die verändert werden sollen.

\subsection{\texttt{edit\_library} -- Bibliotheken verwalten}
\label{subsec:command_list:edit_library}

\begin{verbatim}
edit_library(action: str, name: str, people: Iterable[int] = (),
books: Iterable[int] = (), pay_required: bool = None) -> set
    Perform the given action on the Library with the given name

    'delete' will remove the reference to the library from all given people
        and books (setting their library to 'main'),
        but not actually delete the Library itself
        ``people`` and ``books`` are ignored in this case
    'add' will add the Library to all given people and set the library
        of all the given books to the specified one
    'remove' will remove the reference to the given Library in the given people
        and set the library of the given books to 'main'

    ``name`` is the name of the Library to modify
    ``people`` is an iterable of the IDs of the people to modify
    ``books`` is an iterable of IDs of the books to modify
    ``pay_required`` will set the Library's payment requirement 
to itself if not None

    return a set of strings describing encountered errors
\end{verbatim}

Parameter:

\begin{tabular}{|p{0.2\textwidth}|p{0.2\textwidth}|p{0.6\textwidth}|}
Name & Wertart & Bemerkung\\
\hline
action & Text & Die auszuführende Aktion (wird gleich näher erläutert) \\
name & Text & Der Name der Bibliothek \\
people & Liste aus ganzen Zahlen & Die Schülerausweisnummern der Schüler \\
books & Liste aus ganzen Zahlen & Die IDs der Bücher \\
pay\_required & Wahrheitswert & Ob Schüler bezahlt haben müssen, um aus der Bibliothek auszuleihen. Wenn nichts angegeben wird, passiert nichts.
\end{tabular}

\paragraph{``add''}
Mit dieser Aktion werden alle angegebenen Bücher in die angegebene Bibliothek gebracht, d.h. sie sind ab sofort in der angegebenen Bibliothek, egal, in welcher Bibliothek sie vorher waren. Außerdem erhalten alle angegebenen Schüler Zugang zur angegebenen Bibliothek.

\hinweis{Wenn gleiche Bücher öfters die Bibliothek wechseln müssen, kann es sinnvoll sein, diese in eine Gruppe zusammenzufassen. Siehe dazu auch \linkandref{subsec:command_list:activate_group}{\command{activate\_group}}}

\paragraph{``remove''}
Mit dieser Aktion werden alle angegebenen Bücher aus der angegebenen Bibliothek entfernt und in die Bibliothek ``main'' gebracht. Alle angegebenen Schüler verlieren Zugang zur angegebenen Bibliothek.

\hinweis{Wenn die Bücher in eine andere Bibliothek verschoben werden sollen, kann dies direkt passieren, indem die Zielbibliothek angegeben und als Aktion ``hinzufügen'' gewählt wird.}

\paragraph{``delete''}
Mit dieser Aktion werden alle sich in der Bibliothek befindlichen Bücher zurück in die Bibliothek ``main'' gebracht und alle Schüler, die Zugang zur Bibliothek hatten, verlieren diesen. Die Parameter ``people'' und ``books'' werden in diesem Fall ignoriert.

\hinweis{Die Bibliothek selbst bleibt bestehen. Daher eignet sich diese Aktion besonders für Leseclubs, sodass nach Ende nichts, was später für unerwartete Probleme sorgen könnte, übersehen wird.}

\subsection{\texttt{edit\_group} -- Gruppen verwalten}
\label{subsec:command_list:edit_group}

\begin{verbatim}
edit_group(action: str, name: str, books: Iterable[int]) -> set
    Perform the given action on the specified Group

    'delete' will remove the reference to the grou from all given books,
        but not actually delete the Group itself
        ``books`` is ignored in this case
    'add' will add the Group to all given books
    'remove' will remove the reference to the given Group in the given books

    ``name`` is the name of the Group to modify
    ``books`` is an iterable of the IDs of the books to modify

    return a set of strings describing encountered errors
\end{verbatim}

Parameter:

\begin{tabular}{|p{0.2\textwidth}|p{0.2\textwidth}|p{0.6\textwidth}|}
Name & Wertart & Bemerkung\\
\hline
action & Text & Die auszuführende Aktion (wird gleich näher eräutert) \\
name & Text & Der Name der Bibliothek \\
books & Liste aus ganzen Zahlen & Die IDs der Bücher
\end{tabular}

\paragraph{``add''}
Mit dieser Aktion werden alle angegebenen Bücher der angegebenen Gruppe hinzugefügt, d.h. Gruppenaktivationen mit der angegebenen Gruppe sprechen diese Bücher ab sofort an.

\paragraph{``remove''}
Mit dieser Aktion werden alle angegebenen Bücher aus der angegebenen Gruppe entfernt, d.h. Gruppenaktivationen mit der angegebenen Gruppe sprechen diese Bücher ab sofort nicht mehr an.

\paragraph{``delete''}
Mit dieser Aktion werden alle Bücher, die Teil der angegebenen Gruppe waren, aus der angegebenen Gruppe entfernt. Das ``Bücher''-Eingabefeld wird in diesem Fall ignoriert.

\hinweis{Ich persönlich sehe wenig praktischen Nutzen für diese Aktion. Aus Symmetriegründen zur Bibliothek wird sie jedoch angeboten.}

\subsection{\texttt{edit\_member} -- Mitglieder verwalten}
\label{subsec:command_list:edit_member}

\begin{verbatim}
edit_member(m: v2.models.Member, **kwargs)
    Edit a member.

        Set the attributes of the Member object retrieved to the arguments passed.
        wrapper raises BuchSchlossError on failure
    This function requires authentication in form of
    a `current_password` argument containing the currently
    logged in member's password
\end{verbatim}

Als Parameter werden der Name des zu verwaltenden Mitglieds (erste Stelle oder ``person'') und, nur mit Namen, die Parameter, die auch für \command{new\_member} (s. \ref{subsec:command_list:new_member}) genommen. Es müssen nur die Daten angegeben werden, die verändert werden sollen.

\hinweis{Das Passwort muss separat mit \command{change\_password} (\linkandref{subsec:command_list:change_password}) geändert werden}

\hinweis{Das Einzige, was mit diesem Befehl geändert werden sollte, ist das Level eines Mitglieds}

\subsection{\texttt{change\_password} -- Passwort ändern}
\label{subsec:command_list:change_password}

\begin{verbatim}
change_password(m: v2.models.Member, new\_password: str)
    Change a Member's password and use a new salt

        new_password is the new password of the Member being edited
        Editing requires being Superadministrator or the editee
        If the editee is currently logged in, update
    This function requires authentication in form of
    a `current_password` argument containing the currently
    logged in member's password
\end{verbatim}

Als einziger Parameter wird der der Name des Mitglieds, dessen Passwort geändert werden soll, genommen. \textbf{Danach, bei Aufforderung}, werden das Passwort des derzeit eingeloggten Mitglieds sowie das neue Passwort eingegeben.

\subsection{\texttt{activate\_group} -- Gruppe aktivieren}
\label{subsec:command_list:activate_group}

\begin{verbatim}
activate_group(name: str, src: Iterable[str] = (), dest: str = '')
    Activate or deactivate a group.

    name is the group name
    src is an iterable of origin libraries
    dest is the target library
    if src is a falsely value (empty), the origin is all libraries
    if dest is a falsely value, the destination is 'main'

    return a set of strings describing encountered errors or the number of moved books
\end{verbatim}

Parameter:

\begin{tabular}{|p{0.2\textwidth}|p{0.2\textwidth}|p{0.6\textwidth}|}
Name & Wertart & Bemerkung\\
\hline
name & Text & Der Name der Gruppe \\
src & Liste aus Texten & Die Bibliotheken, aus denen die Bücher genommen werden sollen (leer für alle) \\
dest & Text & Die Bibliothek, in die die Bücher kommen sollen.
\end{tabular}

\subsection{\texttt{view\_book} -- Buch betrachten}
\label{subsec:command_list:view_book}

\begin{verbatim}
view_book(book: v2.models.Book)
    Return data about a book.

    Return a dictionary consisting of the following items:
        - contents of config.BOOK_DATA and id and library attributes
        - groups as a string consisting of group names separated by ;
        - status as a string, one of 'verfügbar', 'entliehen' and 'gelöscht'
        - return_date either '-----' or a string representation of the date
            the book will be returned
        - borrowed_by: '-----' or a representation of the borrowing Person
        - borrowed_by_id: ID of the borrowing Person or None
        - __str__: the string reprsentation
\end{verbatim}

Als einziger Parameter wird die ID des Buchs genommen.

\subsection{\texttt{view\_person} -- Schüler betrachten}
\label{subsec:command_list:view_person}

\begin{verbatim}
view_person(person: v2.models.Person)
    Return data about a Person.

    Return a dict consisting of the following items:
        - id, first_name, last_name, class_ max_borrow, pay_date attributes
        - libraries as a string, individual libraries separated by ;
        - borrows as a tuple of strings "<book> bis <date>";
        - borrow_book_ids: a sequence containing the IDs of the borrowed Books
            in the same order as in borrows
        - __str__ , the string representation
\end{verbatim}

Als einziger Parameter wird die Schülerausweisnummer des Schülers genommen.

\subsection{\texttt{view\_member} -- Mitglied betrachten}
\label{subsec:command_list:view_member}

\begin{verbatim}
view_member(member: v2.models.Member)
    Return information about a member

    Return a dictionary with the member's name and level
\end{verbatim}

Als einziger Parameter wird der Name des Mitglieds genommen.

\subsection{\texttt{view\_borrow} -- Ausleihvorgang betrachten}
\label{subsec:command_list:view_borrow}

\begin{verbatim}
view_borrow(borrow: v2.models.Borrow)
    Return information about a borrow action

    Return a dictionary with the following items:
        - 'person': a string representation of the borrowing Person
        - 'person_id': the ID (id) of the borrowing Person
        - 'book': a string representation of the borrowed Book
        - 'book_id': the ID of the borrowed Book
        - 'return_date': a string representation of the date
            by which the book has to be returned
        - 'is_back': a boolean indicating if the book has been returned
        - '__str__': a string representation of the Borrow
\end{verbatim}

Als einziger Parameter wird die ID des Ausleihvorgangs\footnote{z.Z. kann diese nur über eine Suche eingesehen werden} genommen.

\subsection{\texttt{borrow} -- Ausleihe}
\label{subsec:command_list:borrow}

\begin{verbatim}
borrow(b: v2.models.Book, p: v2.models.Person, time: T.Union[int, float])
    Borrow a book.

    time is the time to borrow in weeks.

    raise an error if
        a) the Person or book does not exist
        b) the person has reached their limit set in max_borrow
        c) the person is not allowed to access the library the book is in
        d) the book is not available
        e) the person has not paid for over 52 weeks and the book's
            Library requires payment
        f) the time exceeds the one allowed to the executing member
        g) the time is <= 0

    level | max weeks borrowing allowed
    1     | 5
    2     | 7
    3     | 10
    4     | 20
\end{verbatim}

Parameter:

\begin{tabular}{|p{0.2\textwidth}|p{0.2\textwidth}|p{0.6\textwidth}|}
Name & Wertart & Bemerkung\\
\hline
b & ganze Zahl & ID des Buches, das ausgeliehen wird \\
p & ganze Zahl & Schülerausweisnummer des ausleihenden Schülers \\
time & Zahl & Zeitraum der Ausleihe in Wochen
\end{tabular}

\subsection{\texttt{restitute} -- Rückgabe}
\label{subsec:command_list:restitute}

\begin{verbatim}
restitute(book: v2.models.Book, person: v2.models.Person = None,
person_match: bool = True)
    Restitute a book. Return None.

    raise a BuchSchlossError if:
        a) the book or person doesn't exist OR
        b) the book is not borrowed OR
        c) the person has not borrowed the book and `person_match` is True
\end{verbatim}

Parameter:

\begin{tabular}{|p{0.2\textwidth}|p{0.2\textwidth}|p{0.6\textwidth}|}
Name & Wertart & Bemerkung\\
\hline
book & ganze Zahl & Das zurückgegebene Buch \\
person & ganze Zahl & Die Schülerausweisnummer des zurückgebenden Schülers, falls bekannt \\
person\_match & Wahrheitswert & ``False'', wenn kein Schüler angegeben wurde.
\end{tabular}

\subsection{\texttt{search} -- Suche}
\label{subsec:command_list:search}

\begin{verbatim}
search(o: Type[v2.models.Model], condition: Tuple[str, str, str] = None,
*complex_params: 'ComplexSearch', complex_action: str = 'or',
_in_=None, _eq_=None)
    Search for objects.

        `condition` is a tuple of the form (<a>, <op>, <b>)
            with <op> being a logical operation ("and" or "or") and <a>
                and <b> in that case being condition tuples
            or a comparison operation ("contains", "eq", "ne", "gt", "ge", 
                "lt" or "le")
                in which case <a> is a (possibly dotted) string corresponding
                to the attribute name and <b> is the value to compare to.
        `complex_params` is a sequence of ComplexSearch instances to apply after
            executing the SQL SELECT
        `complex_action` is "and" or "or" and specifies how to handle multiple
            complex cases. If finer granularity is needed, it can be achieved with
            bitwise operators, providing bools are used.

        Note: for `condition`, there is no "not" available.
            Use the inverse comparision operator instead

        This function is compatible with its predecessor and will forward
            `complex_params`, `_in_`, `_eq_` and `condition` (as `kind` argument)
            to the old function.
\end{verbatim}

Parameter:

\begin{tabular}{|p{0.2\textwidth}|p{0.2\textwidth}|p{0.5\textwidth}|p{0.1\textwidth}|}
Name & Wertart & Bemerkung \\
\hline
o & Spezielle Variable & ``Book''; ``Person'', ``Borrow''; ``Member''; ``Group''; ``Library'', je nachdem, wonach gesucht werden soll\\
condition & Liste & Die Bedingung(en), mit der/denen gesucht werden soll. Das Format ist im Hilfstext erläutert\\
\end{tabular}
Die weiteren Parameter sind nicht relevant.


\subsection{\texttt{list} -- Liste}
\label{subsec:command_list:list}

\begin{verbatim}
<lambda> lambda x
\end{verbatim}

Als einziger Parameter wird eine Abfolge (wie z.B. von einer Suche zurückgegeben) genommen.

\subsection{\texttt{print} -- Anzeige}
\label{subsec:command_list:command}

\begin{verbatim}
pprint(object, stream=None, indent=1, width=80, depth=None, *, compact=False)
    Pretty-print a Python object to a stream [default is sys.stdout].
\end{verbatim}

Parameter:

\begin{tabular}{|p{0.2\textwidth}|p{0.2\textwidth}|p{0.6\textwidth}|}
Name & Wertart & Bemerkung\\
\hline
object & egal & was angezeigt werden soll \\
indent & ganze Zahl & wie weit jeweils eingerückt werden soll \\
width & ganze Zahl & wie breit dei ANzeige maximalsein soll \\
depth & ganze Zahl & wie tief die Anzeige gehen soll \\
TODO compact???
\end{tabular}

in einem Gro\ss teil der Fälle sollte nur der erste Parameter angegeben werden müssen.

\paragraph{Beispiel}
\command{> print (1,2,3,(4,5,6)) width=7}
\begin{verbatim}
(1,
 2,
 3,
 (4,
  5,
  6))
\end{verbatim}

\subsection{\texttt{attr} -- Attribute}
\label{subsec:command_list:attr}

\begin{verbatim}
getattr(...)
    getattr(object, name[, default]) -> value

    Get a named attribute from an object; getattr(x, 'y') is equivalent to x.y.
    When a default argument is given, it is returned when the attribute doesn't
    exist; without it, an exception is raised in that case.
\end{verbatim}

Parameter:

\begin{tabular}{|p{0.2\textwidth}|p{0.2\textwidth}|p{0.6\textwidth}|}
Name & Wertart & Bemerkung\\
\hline
object & egal & Das Objekt \\
name & Text & Der Name des Attributs \\
default & egal & Rückgabewert, falls es kein Attribut mit dem Namen gibt
\end{tabular}

\paragraph{Beispiele}
\begin{itemize}
\item \command{> attr <Buch> isbn}
\item \command{> attr <Buch> id}
\end{itemize}
wenn \texttt{Buch} eine ein Buch darstellende Variable ist.

\subsection{\texttt{item} -- Elemente}
\label{subsec:command_list:item}

\begin{verbatim}
getitem(a, b, /)
    Same as a[b].
\end{verbatim}

Parameter:

\begin{tabular}{|p{0.2\textwidth}|p{0.2\textwidth}|p{0.6\textwidth}|}
Name & Wertart & Bemerkung\\
\hline
a & egal & Das Objekt \\
b & egal & Wert des Objekts \\
\end{tabular}

\paragraph{Beispiel}

\begin{verbatim}
----- -> view_book 1
{'__str__': 'Buch[1]"TestTitle"',
 'author': 'TestAuthor',

[ ... ]

 'series': '',
 'shelf': 'A1',
 'status': 'entliehen',
 'title': 'TestTitle',
 'year': '0'}
----- -> item <> shelf
'A1'
\end{verbatim}

\subsection{\texttt{exit} -- Verlassen}
\label{subsec:command_list:exit}

\begin{verbatim}
throw(*si, **nk) method of builtins.type instance
\end{verbatim}

Verlässt, egal welche Parameter gegeben wurden. Falls ein unerwarteter Fehler aufgetreten ist, wird vor dem Verlassen gefragt, ob eine Fehlermeldung per E-Mail verschickt werden soll.

Alternativ kann auch mit \command{Strg-C} verlassen werden. in diesem Fall gibt es keine Frage, falls unerwartete Fehler aufgetreten sind.

\subsection{\texttt{set} -- Variable definieren}
\label{subsec:command_list:set}

\begin{verbatim}
setvar(name, value)
\end{verbatim}

Parameter:

\begin{tabular}{|p{0.2\textwidth}|p{0.2\textwidth}|p{0.6\textwidth}|}
Name & Wertart & Bemerkung\\
\hline
name & Text & der Name der Variable \\
value & egal & der Wert der Variable
\end{tabular}

\paragraph{Beispiel}
\command{> set x 5}

\subsection{\texttt{vars} - Variablen}
\label{subsec:command_list:vars}

\begin{verbatim}
lsvars()
\end{verbatim}

keine Parameter

Zeigt alle derzeit definierten Variablen an.

\paragraph{Beispiel}
\begin{verbatim}
----- -> vars
Person = <Model: Person>
Library = <Model: Library>
Book = <Model: Book>
Group = <Model: Group>
Borrow = <Model: Borrow>
Member = <Model: Member>
Misc = <Model: Misc>
x = 5
\end{verbatim}

\subsection{\texttt{foreach} -- Schleifen}
\label{subsec:command_list:foreach}

\begin{verbatim}
foreach(iterable)
    Iterate over the given iterable.

    execute the following commands for each elemen tin the iterable
    the element is accessible as <> in the first instruction
\end{verbatim}

Mit dem \command{foreach}-Befehl kann man für viele Werte hintereinander die gleichen Befehle ausführen. Als Parameter wird eine Liste oder der von einer Suche zurückgegebene Wert genommen. Dann werden die auszuführenden Befehlen eingegeben, bis eine Leerzeile eingegeben wird.

Im ersten weiteren Befehl ist das Element aus der Liste bzw. dem Suchwert mit \texttt{<>} abrufbar.

\paragraph{Beispiel}

Der Schleifenbefehl kann z.B. benutzt werden, um alle bücher einer Serie in ein anderes Regal zu bewegen. Hier werden die dazu notwendigen Eingaben schrittweise erklärt.

\begin{itemize}
\item Zunächst suchen wir die nötigen Bücher:\\ \command{> search <Book> "('series', 'eq', 'Die drei ???')"}
\item Dann Beginnen wir die Schleife: \command{> foreach <>}
\item Jetzt geben wir ein, was mit den gefundenen Büchern passieren soll
\begin{itemize}
\item Zuerst holen wir die ID: \command{... attr <> id}
\item Dann ändern wir das Regal: \command{... edit\_book <> shelf=C4}
\item Zuletzt geben wir eine leere Zeile ein, um die Schleife zu starten
\end{itemize}
\item Jetzt wird das Regal für jedes Buch geändert. Wir sehen die IDs der Bücher.
\end{itemize}

\end{document}
